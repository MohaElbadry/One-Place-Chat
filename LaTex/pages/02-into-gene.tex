\chapter*{\Huge General Introduction}
\addcontentsline{toc}{section}{\textbf{Introduction}}

As part of our engineering studies at \textbf{École Normale Supérieure de l'Enseignement Technique de Mohammedia (ENSET-M)}, I undertook an internship at \textbf{Zenika}, a leading technology consulting company. This internship provided an exceptional opportunity to apply our academic knowledge in artificial intelligence and software development to real-world challenges while working alongside experienced professionals.\\

The modern software landscape is characterized by an ever-growing ecosystem of APIs that power everything from mobile applications to enterprise systems. However, interacting with these APIs traditionally requires technical expertise: understanding endpoint structures, authentication mechanisms, request parameters, and response formats. This technical barrier limits access to powerful tools and services for non-technical users and creates friction even for developers working with unfamiliar APIs.\\

It is within this context that \textbf{One Place Chat} was conceived—a conversational platform designed to democratize API interactions through natural language. Rather than requiring users to learn specific API syntaxes and structures, One Place Chat enables them to describe their intentions in plain English and automatically translates these requests into precise API calls.\\

The innovation of One Place Chat lies in its intelligent three-stage architecture: automatic tool generation from OpenAPI specifications, semantic tool discovery using vector embeddings, and natural language parameter extraction powered by large language models. This combination creates a seamless experience where complex technical operations become as simple as having a conversation.\\

This report details the complete design and development journey of One Place Chat, presenting the architectural decisions, technologies employed, and features implemented. It demonstrates our methodological approach and the practical application of AI technologies to solve real accessibility challenges in software interaction.

\vspace{1cm}
