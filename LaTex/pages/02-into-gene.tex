\chapter*{\Huge Introduction Générale}
\addcontentsline{toc}{section}{\textbf{Introduction}}

Dans le cadre de notre formation d'ingénieurs à \textbf{l'École Normale Supérieure de l'Enseignement Technique de Mohammedia (ENSET-M)}, nous sommes amenés à développer un projet d'innovation à l'issue de notre deuxième année d'études. Cette démarche constitue un élément central de notre parcours académique, permettant de matérialiser les compétences théoriques et pratiques acquises durant notre formation par la conception et la réalisation d'une solution technologique innovante. Le projet d'innovation représente une opportunité exceptionnelle de mettre en application notre expertise en intelligence artificielle, développement logiciel et technologies cloud, tout en explorant des problématiques réelles du monde professionnel.\\

C'est dans ce contexte académique exigeant que nous avons choisi de nous concentrer sur une problématique contemporaine majeure : la complexité croissante de la recherche d'emploi dans le paysage professionnel actuel. Les candidats font face à des défis multiples : multiplication des plateformes de recrutement, exigences techniques accrues des recruteurs, nécessité de personnaliser chaque candidature, et difficulté à maintenir un suivi efficace des démarches entreprises. Face à ces enjeux, l'intelligence artificielle émerge comme une solution prometteuse pour repenser et optimiser l'expérience de recherche d'emploi.\\

C'est dans cette perspective que naît le projet \textbf{JobAI}, une plateforme innovante conçue pour révolutionner l'accompagnement des chercheurs d'emploi. Plutôt que de se contenter d'automatiser des tâches isolées, JobAI propose une approche holistique qui intègre l'intelligence artificielle à chaque étape du parcours de candidature, depuis la création des documents professionnels jusqu'à la postulation automatisée.\\

La valeur ajoutée de JobAI réside dans sa capacité à combiner plusieurs technologies de pointe pour créer un écosystème complet et cohérent. La plateforme exploite les modèles de langage avancés pour générer des \textbf{CV} et \textbf{lettres de motivation personnalisés}, utilise des techniques d'analyse sémantique pour évaluer la compatibilité entre profils et offres d'emploi, et déploie \textbf{un agent intelligent autonome} capable de postuler automatiquement sur LinkedIn selon les préférences définies par l'utilisateur.\\

Ce rapport détaille la démarche complète de conception et de développement de JobAI, en présentant les choix technologiques adoptés, l'architecture mise en place, et les fonctionnalités développées. Il illustre également notre approche méthodologique, inspirée des pratiques agiles, qui nous a permis de mener ce projet avec la rigueur et l'efficacité d'une équipe de développement professionnelle.

\vspace{1cm}




