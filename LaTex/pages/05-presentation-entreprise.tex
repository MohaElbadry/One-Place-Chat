\renewcommand{\chaptername}{Chapter}
\chapter{General Project Context}
\label{chap:general-context}
\vspace{-1.5cm}
\section{Introduction}
This chapter outlines the general framework of the end-of-studies project. It begins with a detailed presentation of the host organization, Zenika, including its history, structure, and areas of expertise. Following this, the chapter introduces the project context, related concepts, and the methodology adopted for the solution's realization, concluding with the project planning.
\vspace{-0.5cm}
\section{The Host Organization: Zenika}

\subsection{Overview of the Company}


\begin{minipage}{0.15\textwidth}
	\includegraphics[height=5em]{LOGOS/zenika.jpeg}
\end{minipage}%
\begin{minipage}{0.8\textwidth}
	Zenika is a leading French technology consultancy and training firm founded in 2006. Specializing in digital transformation, organizational agility, and software architecture, Zenika acts as a bridge between the organic world and the digital world. The company supports its clients—ranging from startups to large enterprises—in their technological evolution by offering high-end services in consulting, training, and custom software development.
	\\
\end{minipage}

\subsection{Technical Fact Sheet}
To provide a concise overview of the organization, the following table summarizes Zenika's key identity and operational metrics.

\begin{table}[H]
\centering
\renewcommand{\arraystretch}{1.5}
\begin{tabular}{|p{5cm}|p{8cm}|}
\hline
\textbf{Characteristic} & \textbf{Description} \\
\hline
\textbf{Company Name} & Zenika \\
\hline
\textbf{Founded} & 2006 \\
\hline
\textbf{Headquarters} & Paris, France \\
\hline
\textbf{Sector} & IT Consulting, Software Development \\
\hline
\textbf{Workforce} & +600 employees (Consultants, Developers...) \\
\hline
\textbf{Global Presence} & France, Canada, Singapore, Morocco \\
\hline
\textbf{Number of Agencies} & 14 (including Paris, Lyon, Casablanca, etc.) \\
\hline
\end{tabular}
\caption{Zenika Technical Fact Sheet}
\label{tab:zenika-fact-sheet}
\end{table}

\subsection{Geographic Presence and History}
Since its inception, Zenika has pursued a strategy of proximity to its clients and consultants. This has led to a rapid geographic expansion.
\begin{itemize}
    \item \textbf{France}: The network includes agencies in major tech hubs such as Paris, Rennes, Nantes, Bordeaux, Lille, Lyon, Grenoble, Clermont-Ferrand, Niort, and Brest.
    \item \textbf{International}: Zenika has expanded beyond French borders with established agencies in Montreal (Canada), Singapore, and Casablanca (Morocco).
\end{itemize}

This distributed structure allows Zenika to combine the agility of local teams with the strength and knowledge base of a global group.

\section{Organizational Structure and Culture}

\subsection{A Consultant-Centric Model}
Zenika's organizational structure is designed to foster autonomy and expertise. The workforce is primarily composed of highly qualified profiles:
\begin{itemize}
    \item \textbf{Consultants}: Experts who integrate into client teams to provide guidance and technical leadership.
    \item \textbf{Trainers}: Professionals who deliver certified training courses to external clients.
    \item \textbf{Developers}: Full-stack engineers capable of building complex solutions from scratch.
\end{itemize}

\subsection{Corporate Vision: "Code the World"}
Zenika's vision extends beyond commercial success; it aims to impact the technological landscape positively. This vision is supported by several internal and external pillars:

\subsubsection{Knowledge Sharing Initiatives}
\begin{itemize}
    \item \textbf{TechnoZaures}: Internal conference days dedicated to technical exchange among employees.
    \item \textbf{Lunch \& Learn}: Informal lunchtime sessions where collaborators present on specific topics.
    \item \textbf{NightClazz}: Evening meetups open to the public, allowing deep dives into specific technologies.
\end{itemize}

\subsubsection{Continuous Learning}
\begin{itemize}
    \item \textbf{Learning Expeditions}: Zenika organizes trips to major tech hubs (like Silicon Valley) and exchanges between agency offices to foster cross-pollination of ideas.
    \item \textbf{Conference Access}: Consultants are encouraged to attend and speak at national and international tech conferences, ensuring they remain at the cutting edge of industry trends.
\end{itemize}

\subsubsection{Community Support}
Zenika is a strong supporter of the Open Source community and local tech ecosystems. The company sponsors numerous agile and technical conferences and hosts community events in its offices.

\section{Services and Strategic Partnerships}

\subsection{Service Offerings}
Zenika's value proposition is built around three main axes:
\begin{enumerate}
    \item \textbf{Consulting}: Auditing existing systems, defining architectural strategies, and guiding digital transformation (Agile, DevOps adoption).
    \item \textbf{Training}: Offering a comprehensive catalog of training courses. Zenika is often an official training partner for major technologies.
    \item \textbf{Realization}: Developing custom software solutions, MVPs, and industrializing products using modern stacks.
\end{enumerate}

\subsection{Strategic Partnerships}
To deliver the best solutions, Zenika maintains strong partnerships with industry leaders. These alliances allow Zenika to offer certified expertise and early access to new technologies.

Key partners include:
\begin{itemize}
    \item \textbf{Cloud \& Infrastructure}: AWS, Google Cloud, Docker, Kubernetes.
    \item \textbf{Data \& Search}: Elastic, Confluent (Kafka), MongoDB.
    \item \textbf{Development Ecosystem}: Spring, GitHub, GitLab.
\end{itemize}

These partnerships reinforce Zenika's capability to handle complex, large-scale projects and provide objective, expert advice on tool selection.